\documentclass[]{article}
\usepackage{lmodern}
\usepackage{amssymb,amsmath}
\usepackage{ifxetex,ifluatex}
\usepackage{fixltx2e} % provides \textsubscript
\ifnum 0\ifxetex 1\fi\ifluatex 1\fi=0 % if pdftex
  \usepackage[T1]{fontenc}
  \usepackage[utf8]{inputenc}
\else % if luatex or xelatex
  \ifxetex
    \usepackage{mathspec}
  \else
    \usepackage{fontspec}
  \fi
  \defaultfontfeatures{Ligatures=TeX,Scale=MatchLowercase}
\fi
% use upquote if available, for straight quotes in verbatim environments
\IfFileExists{upquote.sty}{\usepackage{upquote}}{}
% use microtype if available
\IfFileExists{microtype.sty}{%
\usepackage{microtype}
\UseMicrotypeSet[protrusion]{basicmath} % disable protrusion for tt fonts
}{}
\usepackage[margin=1in]{geometry}
\usepackage{hyperref}
\hypersetup{unicode=true,
            pdftitle={Impacto del consumo de videojuegos de universitarios en su desempeño académico.},
            pdfborder={0 0 0},
            breaklinks=true}
\urlstyle{same}  % don't use monospace font for urls
\usepackage{color}
\usepackage{fancyvrb}
\newcommand{\VerbBar}{|}
\newcommand{\VERB}{\Verb[commandchars=\\\{\}]}
\DefineVerbatimEnvironment{Highlighting}{Verbatim}{commandchars=\\\{\}}
% Add ',fontsize=\small' for more characters per line
\usepackage{framed}
\definecolor{shadecolor}{RGB}{248,248,248}
\newenvironment{Shaded}{\begin{snugshade}}{\end{snugshade}}
\newcommand{\KeywordTok}[1]{\textcolor[rgb]{0.13,0.29,0.53}{\textbf{#1}}}
\newcommand{\DataTypeTok}[1]{\textcolor[rgb]{0.13,0.29,0.53}{#1}}
\newcommand{\DecValTok}[1]{\textcolor[rgb]{0.00,0.00,0.81}{#1}}
\newcommand{\BaseNTok}[1]{\textcolor[rgb]{0.00,0.00,0.81}{#1}}
\newcommand{\FloatTok}[1]{\textcolor[rgb]{0.00,0.00,0.81}{#1}}
\newcommand{\ConstantTok}[1]{\textcolor[rgb]{0.00,0.00,0.00}{#1}}
\newcommand{\CharTok}[1]{\textcolor[rgb]{0.31,0.60,0.02}{#1}}
\newcommand{\SpecialCharTok}[1]{\textcolor[rgb]{0.00,0.00,0.00}{#1}}
\newcommand{\StringTok}[1]{\textcolor[rgb]{0.31,0.60,0.02}{#1}}
\newcommand{\VerbatimStringTok}[1]{\textcolor[rgb]{0.31,0.60,0.02}{#1}}
\newcommand{\SpecialStringTok}[1]{\textcolor[rgb]{0.31,0.60,0.02}{#1}}
\newcommand{\ImportTok}[1]{#1}
\newcommand{\CommentTok}[1]{\textcolor[rgb]{0.56,0.35,0.01}{\textit{#1}}}
\newcommand{\DocumentationTok}[1]{\textcolor[rgb]{0.56,0.35,0.01}{\textbf{\textit{#1}}}}
\newcommand{\AnnotationTok}[1]{\textcolor[rgb]{0.56,0.35,0.01}{\textbf{\textit{#1}}}}
\newcommand{\CommentVarTok}[1]{\textcolor[rgb]{0.56,0.35,0.01}{\textbf{\textit{#1}}}}
\newcommand{\OtherTok}[1]{\textcolor[rgb]{0.56,0.35,0.01}{#1}}
\newcommand{\FunctionTok}[1]{\textcolor[rgb]{0.00,0.00,0.00}{#1}}
\newcommand{\VariableTok}[1]{\textcolor[rgb]{0.00,0.00,0.00}{#1}}
\newcommand{\ControlFlowTok}[1]{\textcolor[rgb]{0.13,0.29,0.53}{\textbf{#1}}}
\newcommand{\OperatorTok}[1]{\textcolor[rgb]{0.81,0.36,0.00}{\textbf{#1}}}
\newcommand{\BuiltInTok}[1]{#1}
\newcommand{\ExtensionTok}[1]{#1}
\newcommand{\PreprocessorTok}[1]{\textcolor[rgb]{0.56,0.35,0.01}{\textit{#1}}}
\newcommand{\AttributeTok}[1]{\textcolor[rgb]{0.77,0.63,0.00}{#1}}
\newcommand{\RegionMarkerTok}[1]{#1}
\newcommand{\InformationTok}[1]{\textcolor[rgb]{0.56,0.35,0.01}{\textbf{\textit{#1}}}}
\newcommand{\WarningTok}[1]{\textcolor[rgb]{0.56,0.35,0.01}{\textbf{\textit{#1}}}}
\newcommand{\AlertTok}[1]{\textcolor[rgb]{0.94,0.16,0.16}{#1}}
\newcommand{\ErrorTok}[1]{\textcolor[rgb]{0.64,0.00,0.00}{\textbf{#1}}}
\newcommand{\NormalTok}[1]{#1}
\usepackage{longtable,booktabs}
\usepackage{graphicx,grffile}
\makeatletter
\def\maxwidth{\ifdim\Gin@nat@width>\linewidth\linewidth\else\Gin@nat@width\fi}
\def\maxheight{\ifdim\Gin@nat@height>\textheight\textheight\else\Gin@nat@height\fi}
\makeatother
% Scale images if necessary, so that they will not overflow the page
% margins by default, and it is still possible to overwrite the defaults
% using explicit options in \includegraphics[width, height, ...]{}
\setkeys{Gin}{width=\maxwidth,height=\maxheight,keepaspectratio}
\IfFileExists{parskip.sty}{%
\usepackage{parskip}
}{% else
\setlength{\parindent}{0pt}
\setlength{\parskip}{6pt plus 2pt minus 1pt}
}
\setlength{\emergencystretch}{3em}  % prevent overfull lines
\providecommand{\tightlist}{%
  \setlength{\itemsep}{0pt}\setlength{\parskip}{0pt}}
\setcounter{secnumdepth}{0}
% Redefines (sub)paragraphs to behave more like sections
\ifx\paragraph\undefined\else
\let\oldparagraph\paragraph
\renewcommand{\paragraph}[1]{\oldparagraph{#1}\mbox{}}
\fi
\ifx\subparagraph\undefined\else
\let\oldsubparagraph\subparagraph
\renewcommand{\subparagraph}[1]{\oldsubparagraph{#1}\mbox{}}
\fi

%%% Use protect on footnotes to avoid problems with footnotes in titles
\let\rmarkdownfootnote\footnote%
\def\footnote{\protect\rmarkdownfootnote}

%%% Change title format to be more compact
\usepackage{titling}

% Create subtitle command for use in maketitle
\providecommand{\subtitle}[1]{
  \posttitle{
    \begin{center}\large#1\end{center}
    }
}

\setlength{\droptitle}{-2em}

  \title{Impacto del consumo de videojuegos de universitarios en su desempeño
académico.}
    \pretitle{\vspace{\droptitle}\centering\huge}
  \posttitle{\par}
    \author{}
    \preauthor{}\postauthor{}
    \date{}
    \predate{}\postdate{}
  

\begin{document}
\maketitle

Sección: 4 / Profesora: Brigida Molina Carabaño

\section{Integrantes:}\label{integrantes}

-Juan Felix Elías García (Líder)

-Jairo David Narro Silva

-Jean Paul Miraval Obregón

-Jorge Luis Acosta Romero

-Paolo Facundo Alatrista Amaya

\section{Segunda Entrega:}\label{segunda-entrega}

\subsection{Introducción}\label{introduccion}

Jugar videojuegos se ha visto desde la opinión pública como una
actividad meramente de entretenimiento y el invertir tiempo en estas
actividades paralelamente a los estudios es vista de mala manera por la
sociedad, asumiendo que esto trae consecuencias negativas y afecta
directamente al desempeño académico; sin embargo, este juicio es
realizado en la mayoría de casos sin sustento alguno. Este estudio tiene
como objetivo y según el análisis que se realice, obtener ese sustento;
o en caso de que el análisis demuestre que es incorrecto, desmentir esta
creencia. Los datos recolectados para este estudio fueron obtenidos a
través de una encuesta enviada a una página de facebook relacionada con
los videojuegos.

\subsection{Población objetivo}\label{poblacion-objetivo}

Estudiantes universitarios limeños que consumen videojuegos.

\subsection{Población muestra}\label{poblacion-muestra}

Estudiantes universitarios limeños pertenecientes a la comunidad gamer
que respondieron la encuesta.

\subsection{Tipo de muestreo}\label{tipo-de-muestreo}

El tipo de muestreo realizado fue un muestreo estratificado, donde los
grupos heterogéneos son las universidades y los miembros con
características en común son los estudiantes gamers.

\subsection{Audiencia de interés}\label{audiencia-de-interes}

La audiencia de interés de este estudio son principalmente los
estudiantes universitarios peruanos que pertenecen a la comunidad gamer
ya que se dice que son estos los que tienen menor rendimiento académico,
por otro lado, están los alumnos que no se dedican al consumo de
videojuegos pero que al escuchar sobre los prejuicios existentes de esto
tienen la curiosidad de saber si son reales o simplemente mitos.

\subsection{Preguntas de la encuesta}\label{preguntas-de-la-encuesta}

\begin{enumerate}
\def\labelenumi{\arabic{enumi}.}
\tightlist
\item
  ¿Que edad tienes?
\item
  ¿A qué universidad perteneces?
\item
  ¿Qué carrera estás estudiando?
\item
  ¿A qué sexo perteneces?
\item
  ¿A qué rubro de la industria de los videojuegos le dedicas tiempo?
\item
  ¿Sientes que te ha afectado alguna vez académicamente el jugar
  videojuegos?
\item
  ¿Has faltado a clases alguna vez por jugar videojuegos?
\item
  ¿Has dejado de cumplir con tus tareas para dedicar ese tiempo al
  consumo de videojuegos?
\item
  ¿Cuantas horas semanalmente consumes videojuegos?
\item
  ¿Cual es tu promedio en ciencias?
\item
  ¿Cual es tu promedio en letras?
\item
  ¿Cual es tu promedio en humanidades?
\end{enumerate}

\begin{longtable}[]{@{}llc@{}}
\toprule
VARIABLES & TIPOS DE VARIABLES & RELAIÓN CON PREGUNTA\tabularnewline
\midrule
\endhead
Edad & Cuantitativa Discreta & 1\tabularnewline
Universidad & Cualitativa Nominal & 2\tabularnewline
Carrera & Cualitativa Nominal & 3\tabularnewline
Sexo & Cualitativa Nominal & 4\tabularnewline
Tipo de consumo & Cualitativa Nominal & 5\tabularnewline
Horas de consumo semanales & Cuantitativa Continua & 9\tabularnewline
Promedio en el área de ciencias & Cualitativa Ordinal &
10\tabularnewline
Promedio en el área de letras & Cualitativa Ordinal & 11\tabularnewline
Promedio en el área de humanidades & Cualitativa Ordinal &
12\tabularnewline
Influencia negativa en las calificaciones & Cualitativa Nominal &
6\tabularnewline
Influencia negativa en la asistencia a clases & Cualitativa Nominal &
7\tabularnewline
Influencia negativa en la productividad & Cualitativa Nominal &
8\tabularnewline
\bottomrule
\end{longtable}

\section{Preguntas Descriptivas:}\label{preguntas-descriptivas}

\begin{enumerate}
\def\labelenumi{\arabic{enumi}.}
\tightlist
\item
  \textbf{¿Cuál es moda, mediana y media de la edad de los encuestados?}
\end{enumerate}

La edad de los encuestados se distribuye de la siguiente forma:

Media:

\begin{verbatim}
## [1] 20.99
\end{verbatim}

Mediana:

\begin{Shaded}
\begin{Highlighting}[]
\NormalTok{Mediana<-}\KeywordTok{round}\NormalTok{(}\KeywordTok{median}\NormalTok{(DF}\OperatorTok{$}\NormalTok{Edad), }\DecValTok{2}\NormalTok{)}
\NormalTok{Mediana}
\end{Highlighting}
\end{Shaded}

\begin{verbatim}
## [1] 20
\end{verbatim}


\end{document}
